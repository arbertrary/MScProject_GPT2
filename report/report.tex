\documentclass[sigconf,natbib=false]{acmart}

%%%%%%%%
% Packages
%%%%%%%%

\usepackage[backend=biber]{biblatex}

% Filter warnings issued by package biblatex starting with "Patching footnotes failed"
% Source: https://tex.stackexchange.com/questions/202988/beamer-patching-footnotes-warning-patching-footnotes-failed-footnote-detectio
\usepackage{silence}
\WarningFilter{biblatex}{Patching footnotes failed}

% For formal tables
\usepackage{booktabs} 

% Hyperref for formatting urls via the \url{} command.
% Should be loaded last, but before cleverref:
\usepackage{hyperref}

% For automatic reference type labelling.
% For example: for a figure with \label{fig:figure1}, \Cref{fig:figure1} will print ``Figure 1''.
% !loaded last due to hyperref!
\usepackage{cleveref}


%%%%%%%%
% Remove copyright
%%%%%%%%
\setcopyright{none}
\settopmatter{printacmref=false}
\acmISBN{} % set this to remove ISBN
\acmDOI{} % set this to remove DOI


%%%%%%%%
% Meta information
%%%%%%%%
\acmConference[]
	{Seminar: Ausgew\"ahlte Themen des Machine Learning}
	{WS \the\year}


%%%%%%%%
% Bibliography sources
%%%%%%%%

% * you can use a remote bibliography from BibSonomy (change 'dmir' to your own username)
%\addbibresource[location=remote]{http://www.bibsonomy.org/bib/user/dmir/myown}

% * or a local file
\addbibresource{bibliography.bib}



\begin{document}

%%%%%%%%
% Front matter
%%%%%%%%

\title{Bericht Masterpraktikum}
%\subtitle{An optional subtitle}

\author{Armin Bernstetter}
\affiliation{%
  \institution{University of W\"urzburg}
  \today
}
% \email{trovato@corporation.com}


\begin{abstract}
This report is a summary of the 
\end{abstract}


%%%%%%%%
% Content
%%%%%%%%

\maketitle

\section{Introduction/Motivation}

\begin{itemize}
	\item dataset erweitern
	\item language model
\end{itemize}


\section{Crawler}

\begin{itemize}
	\item welche seiten kamen infrage (reddit, TA)
	\item wo lagen probleme beim crawlen?
	\item Was ist team andro
	\item wie ging der crawler vor
\end{itemize}

\section{The Dataset}

\begin{itemize}
	\item der bestehende super corpus
	\item Der neue Team Andro Corpus
\end{itemize}

\section{Elmo}

\begin{itemize}
	\item Visualisierte elmo embeddings? TODO
\end{itemize}

\section{Gpt-2}

\begin{itemize}
	\item Welche implementierung und warum?
	\item der blog post mit den gedichten
	\item welche modelle kamen raus
	\item Fragen und Zweifel/Probleme. Undurchsichtige Trainingsskripte usw
	\item sample output und interactive prompt samples
\end{itemize}



\section{What could have gone better? Future Work}

\begin{itemize}
	\item not depend so much on existing repositories/forks 
	\item rather implement own training scripts for gpt2?
	
\end{itemize}


 
%\input{content/introduction}
%\input{content/othercontent}
%\input{content/conclusion}

\printbibliography

\end{document}
